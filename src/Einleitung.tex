\chapter{Einleitung}
\thispagestyle{empty}
\pagenumbering{arabic}

\section{Hintergrund}
Um die Energiewende bewältigen und das von der Bundesregierung angestrebte Ziel der CO$_2$-Neutralität bis zum Jahr 2050 \cite{bundesministerium2016klimaschutzplan} erreichen zu können, muss - neben der Stromversorgung - der Anteil regenerativer Energien in der Wärmeversorgung erhöht werden. Der Einsatz solarthermischer Anlagen in Nah- und Fernwärme stellt eine Möglichkeit dar, eine seit Jahren etablierte, einfache und ausgereifte Technik einzusetzen. Konzepte einer 4. Generation von Wärmenetzen betonen die Bedeutung solarthermischer Anlagen in Kombination mit saisonalen Wärmespeichern in der künftigen Wärmeversorgung \cite{LUND20141}.

Entgegen der Erwartung, dass Länder wie Italien, Portugal oder Spanien verstärkt Solarthermie einsetzen - Regionen mit einem erheblichen Dargebot an solarer Einstrahlung - ist Dänemark das weltweit führende Land im Bereich der solaren Wärmeversorgung. In Dänemark sind seit dem Jahr 1988 insgesamt 109 Solarthermie-Anlagen mit einer Größe jenseits der 1000~m² in Betrieb genommen worden. Die größte Anlage weltweit wurde 2016 in Silkeborg gebaut - die Fläche beträgt 156.694~m². Demgegenüber stehen 18 deutsche Anlagen - die Größte mit einer Fläche von 8.300~m². \cite{SDH2019} 

Der Vergleich mit Dänemark zeigt, dass die Solarthermie in einem Land mit ähnlichen Umweltbedingungen wirtschaftlich eingesetzt werden kann. Ziel dieser Arbeit ist die Untersuchung der Einsatzmöglichkeiten von Solarthermie in der norddeutschen Wärmeversorgung. Insbesondere soll untersucht werden, ob der Einsatz solarthermischer Anlagen in einem durch \ac{KWK} dominiertem Wärmesystem - in Deutschland üblich - wirtschaftlich eingesetzt werden kann. Darüber hinaus soll untersucht werden, welchen Einfluss die Solarthermie auf den Betrieb eines solchen Systems hat. 

\section{Methodik}
Um die Frage nach der Wirtschaftlichkeit solarthermischer Anlagen beantworten zu können, werden im Rahmen dieser Arbeit Modelle von Wärmeversorgungssystemen einer Einsatzoptimierung mittels linearer Programmierung unterzogen. Nach einer kurzen Zusammenfassung des aktuellen Stands der Wissenschaft, der den Leser in die Lage versetzen soll diese Arbeit in einen größeren Kontext einzuordnen, werden in dem folgenden Abschnitt zunächst die allgemein notwendigen Grundlagen zum Verständnis der Arbeit erläutert. Diese beinhalten die Definition von Wirkungsgraden, die zur Abbildung der verwendeten Technologien verwendet werden, eine Übersicht der genutzten Methoden zur wirtschaftlichen Bewertung und eine Erläuterung des eingesetzten Optimierungsverfahrens.

Aufgrund der besonderen Stellung, die Solarthermie in dieser Arbeit einnimmt, ist das daran anschließende Kapitel allein den Grundlagen der solarthermischen Wärmebereitstellung gewidmet. Es wird vor allem darauf eingegangen, wie der Ertrag solarthermischer Anlagen über die gemessene Sonneneinstrahlung bestimmt werden kann. Außerdem werden unterschiedliche Konzepte der solarthermischen Wärmebereitstellung vorgestellt, von denen drei für detaillierte Untersuchungen ausgewählt werden. Auf eine genaue Beschreibung der Funktionsweise und Besonderheiten einzelner solarthermischer Kollektoren ist verzichtet worden, da diese im Rahmen dieser Arbeit nicht relevant ist. 

Im darauf folgenden Kapitel wird die Modellierung der technischen Anlagen beschrieben. Dieser Abschnitt bespricht, welches Verhalten die verwendeten Technologien unter Teillast zeigen und wie diese entsprechend zur Einbindung in die Einsatzoptimierung zu linearisieren sind. Bei der Solarthermie wird an dieser Stelle untersucht, wie sich das Betriebsverhalten mehrerer Kollektoren von dem Verhalten eines einzelnen unterscheidet, um zu entscheiden, wie der Solarthermie-Ertrag in der Optimierung behandelt werden soll.

Bevor die Arbeit in einer Diskussion der erzielten Ergebnisse und einem Ausblick auf weiterführende Arbeiten schließt, wird in Kapitel \ref{chapter: Techno-ökonomische Betriebsmodelle} die techno-ökonomische Optimierung dargelegt. Es wird dargestellt, welche Simulationssoftware zur Optimierung eingesetzt wurde und wie die zu untersuchenden Energiesysteme in dieser Software abgebildet worden sind. Darüber hinaus werden alle Randparameter (Wetterdaten, Preiszeitreihen, etc.) besprochen, die im Rahmen dieser Arbeit verwendet wurden. Außerdem werden in diesem Kapitel für alle Energiesysteme die Ergebnisse der Optimierung dargestellt.

\section{Stand der Wissenschaft}
Im Jahr 2014 ist von \citet{LUND20141} ein Konzept für eine 4. Generation von Wärmenetzen (\ac{4GDH}) definiert worden. Darin wird hervorgehoben, dass zukünftige Wärmenetze einige Bedingungen erfüllen müssen, um nachhaltige Energiesysteme realisieren zu können. Einer von fünf genannten Kernpunkten ist die Möglichkeit erneuerbare Energien, wie beispielsweise Solar- oder Geothermie, integrieren zu können.

Aufgrund des nicht übereinstimmenden Bedarfs an Wärme und dem solaren Dargebot, welches hauptsächlich im Sommer vorliegt, ist die Verwendungen saisonaler Speicher zu empfehlen. Dies ist von \citet{CARPANETO2015714} in ihrer Arbeit über die Integration von Solarthermie in ein kleines norditalienisches Wärmenetz hervorgehoben worden.

Zunächst ist Einbringung von solarthermischer Wärme primär für relativ kleine Wärmenetze untersucht worden \cite{CARPANETO2015714, PAKERE2018549, JOLY2017865}. Eine Arbeit, die den Einsatz von Solarthermie für ein etwas größeres, aber mit maximal 16~MW Heizlast immer noch relativ kleines Wärmenetz untersucht, ist von \citet{WINTERSCHEID2017579} angefertigt worden. Hierin ist die Einbringung einer solarthermischen Anlage und eines Wärmespeichers in ein durch \ac{KWK} versorgtes Wärmenetz untersucht worden. Dies stellt in Deutschland die dominierende Art der Wärmeversorgung dar \cite{WINTERSCHEID2017579}. In einem durch das Bundesministerium für Wirtschaft und Energie geförderten Forschungsvorhaben zu Solar-\ac{KWK}-Systemen sind verschiedene Modelle unter variierenden Randbedingungen simuliert worden. Es wurde gezeigt, dass Solarthermie ohne wirtschaftliche Verluste in die Wärmeversorgung eingebracht werden kann \cite{berberich2015solar}. Gleichzeitig lässt der Bericht jedoch offen, welchen Beitrag Solar-\ac{KWK}-Systeme zukünftig in der Wärmeversorgung leisten werden.

Eine interessante Alternative zur herkömmlichen Solarthermie stellt eine Kombination aus \ac{PV} und \ac{P2H}, vorzugsweise Wärmepumpen, dar. Diese Variante der Einbringung von Sonnenenergie in die Wärmeversorgung ist jüngst von \citet{GRAVELSINS2019} untersucht worden. Sie haben gezeigt, dass in dem modellierten Energiesystem 47\% des gewonnen \ac{PV}-Stroms zum Betrieb der Wärmepumpe genutzt werden und somit die Flexibilität der \ac{PV}-Anlage vergrößert wird. Fallende \ac{PV}-Preise, die in den kommenden Jahren erwartet werden \cite{Vartiainen2019}, machen dieses Konzept für zukünftige Untersuchungen interessant.

Dieser kurze Überblick über den Stand der Wissenschaft zur Einbringung von Solarthermie in die Wärmeversorgung zeigt, dass in diesem Bereich einige Forschungsvorhaben durchgeführt wurden. Diese bezogen sich jedoch überwiegend auf relativ kleine Wärmenetze. Die Integration von Solarthermie in einem \ac{KWK} basierten System - wie sie in Deutschland typisch sind - ist in Ansätzen untersucht worden, die Wirtschaftlichkeit solcher Solar-\ac{KWK}-Systemen in einer künftigen Energieversorgung ist weiter zu untersuchen.

Diese Arbeit beschränkt sich, vor allem aufgrund der zeitlichen Begrenzung, auf die Einbringung von Solarthermie in ein größeres Wärmenetz mit einer Heizlast von maximal ca. 190~MW und untersucht, wie wirtschaftlich solche Anlagen sind. Der Aspekt der Wirtschaftlichkeit unter variierenden Randbedingungen wird nicht weiter betrachtet.
