\chapter*{Zusammenfassung}
Wärmeversorgungssysteme nachhaltig zu gestalten ist durch die Integration regenerativer Energien möglich. Mit der Solarthermie steht hierbei eine Technologie zur Verfügung, die bereits seit Jahren in Dänemark erfolgreich zur Wärmeversorgung eingesetzt wird. Das Ziel der vorliegenden Arbeit ist die Einsatzmöglichkeit solarthermischer Anlagen bezüglich ihrer Wirtschaftlichkeit in deutschen Wärmeversorgungsnetzen zu untersuchen.

Zu diesem Zweck ist ein Kraft-Wärme-Kopplung dominiertes Versorgungssystem - das Referenzsystem - um drei Konzepte der solarthermischen Wärmebereitstellung erweitert worden. Bei dem ersten Konzept handelt es sich um ein einfaches Solarthermie-Konzept, bestehend aus einem Kollektorfeld und saisonalem Wärmespeicher. Das zweite Konzept verwendet zusätzlich eine Wärmepumpe und einen Kurzzeitspeicher, um ein Wärmesystem mit erhöhtem Power-to-Heat-Anteil zu betrachten. Schließlich wird mit dem dritten System ein alternatives Konzept der solaren Wärmeerzeugung untersucht - eine Kombination aus Photovoltaik und Wärmepumpen. 

Alle Konzepte werden jeweils mit drei unterschiedlichen Kollektorflächen und Speicherkapazitäten modelliert. Die Energiesysteme werden zunächst auf Grundlage einer Einsatzoptimierung hinsichtlich ihres Kapitalwerts mit dem Referenzsystem und untereinander verglichen. Daran anschließend werden die drei wirtschaftlichsten Konzepte einer Detailanalyse unterzogen, um den Einfluss der Solarthermie auf den Betrieb des übrigen Versorgungssystems zu untersuchen. Abschließend werden Einflussanalysen durchgeführt, die den Beitrag einzelner Komponenten auf das Ergebnis der Einsatzoptimierung untersuchen.

Es zeigt sich zunächst, dass durch die Verwendung solarthermischer Anlagen der Kapitalwert des Referenzsystems gesteigert werden kann. Das einfache Solarthermie-Konzept ist hierbei aufgrund der geringen Gesamtkosten als das wirtschaftlich attraktivste Solarthermie-Konzept einzustufen. Das Photovoltaik-Konzept erreicht, trotz eines ähnlichen Erlöses, aufgrund der hohen Investitionskosten für die Wärmepumpen einen niedrigeren Kapitalwert. 

Darüber hinaus haben die Detail- und Einflussanalysen gezeigt, dass der Betrieb einer solarthermischen Anlage ohne die Verwendung eines saisonalen Wärmespeichers unwirtschaftlich ist. Ohne die Möglichkeit der Speicherung ist eine vollständige Nutzung der Solarthermie nicht erreichbar. Zusätzlich hat sich ergeben, dass das Ladeverhalten des Wärmespeichers - auch bei einer hohen Sonneneinstrahlung - bei aktuellen Randbedingungen hauptsächlich von den Strompreisen und somit vom Betrieb der Kraft-Wärme-Kopplung abhängt. Schließlich zeigt die Einflussanalyse, dass bei einem System ohne Wärmespeicher das Hinzufügen eines solchen zu der größten Kapitalwertsteigerung führt.
