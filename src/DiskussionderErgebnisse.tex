\chapter{Diskussion der Ergebnisse}\label{chapter: Diskussion der Ergebnisse}
\thispagestyle{empty}
Folgendes Kapitel wird die Ergebnisse dieser Arbeit zusammenfassen und eine Einschätzung über die Belastbarkeit der Resultate geben. Zunächst wird im ersten Abschnitt dargelegt, welche Schlussfolgerungen dieser Arbeit zu entnehmen sind. Dafür werden die Ergebnisse der Einsatzoptimierung noch einmal vergleichend gegenübergestellt. Daran anschließend folgt eine kritische Betrachtung dieser Arbeit, in der die Belastbarkeit der erzielten Ergebnisse besprochen wird. Abschließend ist in einem Ausblick dargestellt, was im Rahmen dieser Arbeit nicht berücksichtigt ist und stellt somit mögliche Ansatzpunkte für anschließende Arbeiten hervor.

\section{Schlussfolgerungen}\label{section: Schlussfolgerungen}
Die vorliegende Arbeit untersucht die Einsatzmöglichkeiten drei unterschiedlicher solarthermischer Konzepte zur Wärmeversorgung in einem bestehenden Versorgungssystem für den norddeutschen Raum. Zur Bewertung der drei Solarthermie-Konzepte ist zunächst ein Referenzsystem, bestehend aus einer \ac{KWK}-Anlage, einem \acl{SLK} und \acl{EHK}, erstellt worden, welches um die zu untersuchenden Konzepte erweitert wird. Bei dem ersten Solarthermie-Konzept handelt es sich um die Erweiterung des Systems um ein Kollektorfeld und einen saisonalen Wärmespeicher. Das zweite Konzept bindet zusätzlich eine Wärmepumpe und einen Kurzzeitspeicher in das System ein - so soll eine Wärmeversorgung mit erhöhtem \ac{P2H}-Anteil betrachtet werden. Das letzte Konzept untersucht ein System, bei dem die Wärme über eine Kombination aus einer \ac{PV}-Anlage und Wärmepumpen bereitgestellt wird. Bei allen Konzepten werden jeweils drei unterschiedliche Auslegungen für die Kollektorfläche und Speicherkapazität betrachtet. Zur Bestimmung eines ökonomisch optimalen Ergebnisses wird die gemischt-ganzzahlige Optimierung verwendet. Der so ermittelte Erlös des Systems wird, zusammen mit den Investitionskosten, genutzt, um für alle Varianten einen Kapitalwert zu ermitteln.

Es zeigt sich, dass prinzipiell alle betrachteten Konzepte und Szenarien den Kapitalwert des Referenzsystems erhöhen. Gleichzeitig hat der Vergleich aller Kapitalwerte offenbart, dass bei den solarthermischen Konzepten eine Erhöhung der Kollektorfläche und des Speichervolumens zu negativen Kapitalwerten führt. Das betrachtete System aus \ac{PV} und Wärmepumpen weist mit zunehmender Dimensionierung ebenfalls eine Abnahme des Kapitalwerts auf. Im Gegensatz zu den Solarthermie-Systemen nimmt der Kapitalwert des \ac{PV}-Systems jedoch keinen negativen Wert an und stellt somit bei jeder betrachteten Auslegung eine Verbesserung des Referenzsystems dar. Es lässt sich festhalten, dass von den betrachteten Szenarien, jene mit der geringsten Kollektorfläche und Speicherkapazität am wirtschaftlichsten sind. Der Vergleich aller Szenarien zeigt darüber hinaus, dass das einfache Solarthermie-Konzept unter den betrachteten Konzepten den höchsten Kapitalwert erreicht und somit - unter den verwendeten Randbedingungen -  das zu empfehlende Konzept darstellt. 

Das \ac{PV}-Konzept weist wegen der Investitionskosten für die Wärmepumpen - trotz eines höheren Erlöses und niedrigerer Kosten der \ac{PV}-Module - einen geringeren Kapitalwert als das entsprechende Solarthermie-Konzept auf. Die von der \ac{PV}-Anlage bereitgestellte elektrische Energie wird zu 33,5\% für den Betrieb der Wärmepumpe eingesetzt und stellt somit 1,43\% der gesamten Wärmeproduktion, was wiederum ca. 1/3 der Solarthermie entspricht, bereit. Es lässt sich folgern, dass der \ac{PV}-Strom bevorzugt direkt vermarktet wird und nur bei einem geringen Strompreis zur Wärmeproduktion Verwendung findet.

Eine anschließende Einflussanalyse, die den Einfluss einzelner Technologien innerhalb der Konzepte auf das Ergebnis der Optimierung untersucht, zeigt, dass allein die Verwendung eines saisonalen Wärmespeichers den Erlös und Kapitalwert des Systems deutlich erhöht. Die zusätzliche Verwendung einer solarthermischen Anlage steigert den Erlös - sorgt auf Grund der zusätzlichen Investitionskosten jedoch für eine Abnahme des Kapitalwertes. Darüber hinaus hat die Einflussanalyse gezeigt, dass bei Verwendung einer Solarthermie-Anlage ohne Wärmespeicher die Solarthermie nicht gänzlich verwendet werden kann. 

Schließlich kann festgehalten werden, dass durch den Einsatz solarthermischer Wärme - ob direkt durch Solarthermie-Kollektoren oder indirekt durch eine Kombination aus \acl{PV} und Wärmepumpe - das Referenzsystem wirtschaftlicher betrieben werden kann. Das einfache Solarthermie-Konzept stellt dabei unter den untersuchten Konzepten und den zugrunde gelegten Rahmenbedingungen das wirtschaftlich attraktivste Konzept dar. Ein Hinzufügen eines Wärmespeichers ohne Solarthermie hat jedoch die höchste Kapitalwertsteigerung bewirkt - was auf den besseren Betrieb der KWK-Anlage zurückzuführen ist.

\section{Kritische Betrachtung}
Die lineare Optimierung erfordert eine vereinfachte Abbildung aller am Energiesystem beteiligten Komponenten. Im Rahmen dieser Arbeit sind, vor allem aus zeitlichen Gründen, einige Komponenten stärker vereinfacht als andere. Bei der Solarthermie hat die Modellierung ergeben, dass der Ertrag im Preprocessing bestimmt werden kann. Insofern muss die Solarthermie-Anlage nicht linearisiert werden. Mit der GenericCHP-Komponente und dem OffsetTransformer sind zur Abbildung des \ac{GuD} und der \ac{WP} zwei Komponenten gewählt worden, welche die Optimierung zu einem \ac{MILP}-Problem machen, aber eine möglichst genaue Abbildung der Komponenten erlauben. Für den \ac{SLK} und \ac{EHK} ist eine einfachere Modellierung mit einem konstanten Wirkungsgrad gewählt worden, was vor der Erkenntnis, dass der \ac{EHK} kaum und der \ac{SLK} überwiegend bei Nennlast genutzt wird, kaum einen Einfluss auf das Betriebsergebnis hat. Anders verhält es sich bei der Speichermodellierung.

Die verwendeten Wärmespeicher sind, wie der \ac{EHK} und \ac{SLK}, mit einem konstanten Wirkungsgrad auf den austretenden Wärmestrom abgebildet worden. Der Einfluss der Vorlauftemperatur auf das Speicherverhalten oder die Speichertemperatur zur Bestimmung der thermischen Verluste werden in dieser Arbeit nicht berücksichtigt. Über ein Jahr könnten Situationen eintreten, in denen die Speichertemperatur die Vorlauftemperatur des Netzes übersteigt. In diesem Fall kann der Speicher ausschließlich entladen werden. Dies kann mit der vorliegenden Modellierung nicht abgebildet werden.

Darüber hinaus wird in dieser Einsatzoptimierung exemplarisch ein einzelnes Jahr, 2016, betrachtet. Die von dem Betrachtungszeitraum abhängige Einstrahlung und Umgebungstemperaturen haben jedoch einen erheblichen Einfluss auf das Ergebnis der Solarthermie-Anlage sowie \ac{PV}-Anlage und haben somit einen Einfluss auf das Gesamtergebnis der Simulation. Dementsprechend ist bei der Bewertung der Ergebnisse zu berücksichtigen, dass die Umgebungsbedingungen von Jahr zu Jahr variieren und somit das Ergebnis der Einsatzoptimierung bei anderen Betrachtungszeiträumen entsprechend besser oder schlechter ausfallen kann.

Ähnlich verhält es sich bei den angenommenen Kosten für den Strom- und Gasbezug. Diese sind als historische Daten in die Modellierung eingegangen. Es ist jedoch zu bedenken, dass sich diese, wie die Einstrahlung, je nach Betrachtungszeitraum unterscheiden können. Gleiches gilt für die Höhe der Stromabgaben und den CO$_2$-Zertifikatspreis. Im Gegensatz zum realen Betrieb eines Energiesystems, bei dem nur mit kurzfristigen Prognosen gearbeitet werden kann, sind bei der durchgeführten Einsatzoptimierung alle Preiszeitreihen, Wetterdaten und Heizlasten für den gesamten Betrachtungszeitraum bekannt. Die erzielten Ergebnisse der durchgeführten Einsatzoptimierungen sind daher generell als zu hoch einzuschätzen.

\section{Ausblick}
Wie aus Abbildung \ref{fig: Heatmaps Solarthermie} und \ref{figure: Heatmap Photovoltaik} hervorgeht, ist aus wirtschaftlicher Sicht eine Tendenz zu kleineren Anlagendimensionierungen für die Kollektoren und den saisonalen Wärmespeicher zu erkennen. Eine Möglichkeit an diese Arbeit anzuknüpfen ist es daher parallel zur Einsatzoptimierung eine Auslegungsoptimierung der Kollektoren und des Speichers durchzuführen. So kann ein tatsächlich optimales Ergebnis der Einsatzoptimierung erzielt werden. 

Darüber hinaus sind in der vorliegenden Arbeit Solarthermie-Konzepte betrachtet worden, bei denen die Solarthermie direkt in das Wärmeversorgungssystem einspeist. Wie in Kapitel \ref{section: Konzepte} jedoch dargestellt wurde gibt es verschiedene Konzepte, die durch den Einsatz von Wärmepumpen einen Betrieb bei reduzierten Austrittstemperaturen ermöglichen und somit die Effizienz der Solarthermie-Kollektoren erhöhen. Folgende Arbeiten können diese Konzepte untersuchen und klären, ob die zusätzlichen Investitionskosten für die Wärmepumpe eine Effizienzsteigerung der Kollektoren rechtfertigt. 

In diesem Zusammenhang ist es zusätzlich notwendig eine Wasser-Kompressionswärmepumpe abzubilden, die für den Betrieb effizienzsteigender Solarthermie-Konzepte notwendig ist. In dieser Arbeit wird, aus Gründen der zeitlichen Begrenzung, ausschließlich eine Luft-Kompressionswärmepumpe betrachtet. Es ist angenommen worden, dass diese im Winter die Temperaturdifferenz zwischen der Umgebungs- und Vorlauftemperatur zu jedem Zeitpunkt überbrücken kann, dies ist jedoch zu bezweifeln - immerhin kann die Temperaturdifferenz im Winter über 120°C betragen.

Abschließend sollten folgende Arbeiten den Einfluss variierender Energiekosten sowie Investitionskosten untersuchen. Es sollte untersucht werden, welchen Einfluss zunehmende Gas- und CO$_2$-Zertifikatspreise auf die optimale Anlagendimensionierung der Solarthermie-Kollektoren und den Betrieb der \ac{KWK}-Anlage hat, da anzunehmen ist, dass durch zunehmende
Energiekosten höhere Kollektorflächen wirtschaftlich werden.
